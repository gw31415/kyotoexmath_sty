\documentclass[uplatex,b5paper]{jsarticle}
\usepackage{bxpapersize}
\usepackage{kyotoexmath}

\setscoresum{60}
\title{も し 模 試}
\setreadme{
  \begin{enumerate}
    \item カンニングは処す.
    \item 問題は全部で6問ある.
    \item 解答用紙は事前に用意すること.
    \item 電卓等の使用は禁止する.
    \item 解答のための下書き, 計算などは, 計算用紙を用意して書いてもよい.
    \item 解答に関係のないことを書いた答案は無効にすることがある.
    \item 途中経過は加点対象にすることがある.
    \item 解答は分断を避けること.
    \item 問題冊子は持ち帰ってもよい.
  \end{enumerate}
}
\begin{document}
\maketitle

\begin{prob}{10}
  問題
\end{prob}
\begin{prob}{10}
  問題
\end{prob}
\begin{prob}{10}
  問題
\end{prob}
\begin{prob}{10}
  問題
\end{prob}
\begin{prob}{10}
  問題
\end{prob}
\begin{prob}{10}
  問題
\end{prob}
\probends
\end{document}
